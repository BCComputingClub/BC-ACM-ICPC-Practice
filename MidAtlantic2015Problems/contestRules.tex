Welcome to the
\ifPRACTICE
 practice round for the
\fi
2015 ICPC Mid-Atlantic Regional.  Before you start the
contest, please take the time to review the following:

\subsection*{The Contest}

\newcounter{savedEnumi}

\begin{enumerate} 
\ifPRACTICE
\item There is one (1) practice problem.  Please submit solutions or
  request clarifications {\bfseries for this problem only.}  Unless
  you have a real question about the problem, please submit at most
  one clarification request, and at most two runs.  It is important
  that everyone have a chance to see how the system works. Even if you
  do not solve the practice problem, you should submit once just to
  practice with the system.

\item Before completing the practice problem, please
  read all of the notes listed here.  They are designed to help you
  solve the problems during the contest.

\else
\item There are eight (8) problems in the packet, labeled A--H.
  These problems are NOT sorted by difficulty.  As a
  team's solution is judged correct, the team will be awarded a
  balloon.  The balloon colors are as follows:
\begin{center}
\begin{tabular}{|c|c|c|}
  \hline
{\bfseries Problem}&
{\bfseries Problem Name}&
{\bfseries Balloon Color}\\
\hline
\hline
A&
Positive Con Sequences &
Orange \\
\hline
B&
Refract Facts &
Green \\
\hline
C &
Hounded by Indecision &
Silver \\
\hline
D&
Avoiding an Arrrgument   &
Pink \\
\hline
E&
Kinfolk &
Red \\
\hline
F&
Of the Children &
Yellow \\
\hline
G&
Talking About Numbers &
Black \\
\hline
H &
The Scheming Gardener &
Purple \\
\hline
\end{tabular}
\end{center}

%%% Local Variables: 
%%% mode: latex
%%% End: 

\fi


\item The winning team is the one that successfully completes the most
    problems in the time allowed. 

    If teams are tied with the same number of problems solved, the tie
  is broken in favor of the team with the fewest penalty points. 
  For each problem
  {\em solved correctly}, penalty points are charged as the sum of

    \begin{itemize}
    \item the number of minutes elapsed since the start of the contest
       to when the successful submission was made, and
    \item 20 points for each incorrect submission prior to the
    successful one.
    \end{itemize}
  No penalty points are added for problems that are never
  solved.  
  
\item In the event that you feel a problem statement is ambiguous 
  or incorrect, you may request a clarification.  Read the problem
  carefully before requesting a clarification.  

  If a clarification is issued during the contest, it will be
  broadcast to {\em all} teams.

  If the judges believe that the problem statement is sufficiently
  clear, you will receive the response, ``No response, read problem
  statement.''  If you receive this response, you should read the
  problem description more carefully.  If you still feel there is an
  ambiguity, you will have to be more specific or descriptive of the
  ambiguity you may have found.


\setcounter{savedEnumi}{\value{enumi}}

\end{enumerate}


\subsection*{Submitting}

\begin{enumerate}
\setcounter{enumi}{\value{savedEnumi}}

\item Solutions for problems submitted for judging are called runs.
  Each run will be judged.  
  

\item
  The judges will respond to your submission with one of the following
  responses.

\begin{center}
\begin{tabular}{|c|p{4in}|}
  \hline
{\bfseries Response} & {\bfseries Explanation}\\
\hline
\hline
{\bfseries Yes} & Your submission has been judged correct.\\
{\bfseries Wrong Answer} & Your submission generated output that is
not correct. %or is incomplete.
\\
{\bfseries Output Format Error} & Your submission's output is not in
the correct format or is misspelled.\\
{\bfseries Incomplete Output} & Your submission did not produce all of the required output.\\
{\bfseries Excessive Output} & Your submission generated output in
addition to or instead of what is required.\\
{\bfseries Compilation Error} & Your submission failed to compile.\\
{\bfseries Run-Time Error} & Your submission experienced a run-time error.\\
{\bfseries Time-Limit Exceeded} & Your submission did not solve the
judges' test data within 30 seconds.\\
{\bfseries Other-Contact Staff} & Contact your local site judge for clarification.\\
\hline
\end{tabular}
\end{center}


\item In the event that more than one response is applicable,
  the judges may respond with any of the applicable responses.  For
  example, a program that runs too long but produces incorrect
  output before it is killed might receive either a ``Wrong Answer''
  or a ``Time-Limit Exceeded'' response.  A program that crashes before
  completing the test data set might receive either an ``Incomplete
  Output'' or a ``Run-Time Error'' response.




\item \textbf{Do not} request clarifications on
  when a response will be returned.  If you have not received a
  response for a run within 30 minutes of submitting it, {\bfseries
  you may have a runner ask the local site judge to determine the
  cause of the delay.}  

  If, due to unforeseen circumstances, judging for one or more
  problems begins to lag more than 30 minutes behind submissions, a
  clarification announcement will be issued to all teams. This
  announcement will include a change to the 30 minute time period that
  teams are expected to wait before consulting the site judge.
  
\item The submission of code deliberately designed to delay, crash, 
  or otherwise negatively affect the contest itself will be
  considered grounds for immediate disqualification.    




\setcounter{savedEnumi}{\value{enumi}}

\end{enumerate}


\subsection*{Your Programs}

\begin{enumerate}
\setcounter{enumi}{\value{savedEnumi}}

\item Your program must be contained within a single source-code
  file. Java programs should be written in the default (unnamed)
  package, meaning that it should not contain a \texttt{package}
  statement at all.

  Use the filename extension ``{\tt .cpp}'' for C++ program files.
  Use the extension ``{\tt .c}'' for C program files. 
  Use the extension ``{\tt .java}'' for Java program files. 

  Note that all filename extensions are lower case.

\item Your code will be compiled for judging as follows:

    \textbf{C:} \texttt{gcc -O2 -std=gnu99 -static} \textit{yourFileName} \texttt{-lm}

    \textbf{C++:} \texttt{g++ -O2 -std=c++11 -static} \textit{yourFileName}

    \textbf{Java:} \texttt{javac -encoding UTF-8 -sourcepath .} \textit{yourFileName}
    

    For Java, the compiled code will be executed using the command:
    
    \texttt{java -Xss8m -Xmx1024m } \textit{yourClassName}

    

\item All solutions must read from standard input and write to
  standard output.  

  In C this is {\tt scanf}/{\tt printf}, in C++ this
  is {\tt cin}/{\tt cout}, and in Java this is {\tt System.in}/{\tt
  System.out}.  

\item Unless otherwise specified, all lines of program output 
    \begin{itemize}

        \item must be left justified, with no leading blank spaces
          prior to the first non-blank character on that line,

        \item must end with the appropriate line terminator
          (\verb#\n#, \verb#endl#, or \verb#println()#), and

        \item must not contain any blank characters at the end of the
        line, between the final specified output and the line
        terminator.

    \end{itemize}

    You must not print extra lines of output, even if empty, that are
    not specifically required by the problem statement.

\item Unless otherwise specified, all numbers in your output
  should begin with the minus sign ({\tt -}) if negative, followed
  immediately by 1 or more decimal digits.  If the number being
  printed is a floating point number, then the decimal point should
  appear, followed by the appropriate number of decimal digits. 

  For output of real numbers, the number of digits after the decimal
  point will be specified in the problem description (as the
  ``{\em decimal digits of precision}'').

  All floating point numbers printed to a given precision should be
  rounded to the nearest value.  For example, if 2 decimal digits of
  precision is requested, then 0.0152 would be printed as ``0.02'' but
  0.0149 would be printed as ``0.01''.

  In other words, neither scientific notation nor commas will be
  used for numbers, and you should ensure that you use a printing
  technique that rounds to the appropriate precision.

  

\item All input sets used by the judges will follow the input format
  specification found in the problem description. You do not need to
  test for input that violates the input format specified in the problem.
  
\item All lines of program input will end with the
 appropriate line terminator (e.g., a linefeed on Unix/Linux systems,
 a carriage return-linefeed pair on Windows systems).


\item If a problem specifies that
      an input is a floating point number, the input will be presented
      according to the rules stipulated above for output of real
      numbers, except that decimal points and the following digits may
      be omitted for numbers with no non-zero decimal
      portion. Scientific notation will not be used in input sets
      unless a problem explicitly allows it.




\item Every effort has been made to ensure that the compilers
 and run-time environments used by the judges are as similar as
 possible to those that you will use in developing your code.  
 With that said, some differences may exist. It is, in general, 
 your responsibility to write your code in a portable manner
 compliant with the rules and standards of the programming language. 
 You should not rely upon undocumented and non-standard behaviors.

\end{enumerate}


Good luck, and HAVE FUN!!!
